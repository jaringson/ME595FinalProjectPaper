\documentclass[letterpaper, 10 pt, conference]{ieeeconf}  % Comment this line out if you need a paper

%\documentclass[a4paper, 10pt, conference]{ieeeconf}      % Use this line for a4 paper

\IEEEoverridecommandlockouts

\overrideIEEEmargins                                      % Needed to meet printer requirements.

% See the \addtolength command later in the file to balance the column lengths
% on the last page of the document
\usepackage{cite}
% The following packages can be found on http:\\www.ctan.org
\usepackage{graphicx} % for pdf, bitmapped graphics files
%\usepackage{epsfig} % for postscript graphics files
%\usepackage{mathptmx} % assumes new font selection scheme installed
%\usepackage{times} % assumes new font selection scheme installed
\usepackage{amsmath} % assumes amsmath package installed
\usepackage{amssymb}  % assumes amsmath package installed
\usepackage{math}
\usepackage{aircraftshapes}
%\usepackage{natbib}
%\usepackage{standalone}
\usepackage{tikz}
\usetikzlibrary{positioning}
\usetikzlibrary{shapes,arrows}
\tikzstyle{pre}=[<-,>=stealth,thick]
\tikzstyle{post}=[->,>=stealth,thick]
\tikzstyle{prepost}=[<->,>=stealth,thick]
\usetikzlibrary{automata,positioning,graphs,calc}

\usepackage{graphicx}
\usepackage{color}
\usepackage{pgfplots}
\pgfplotsset{compat=1.15}
\usepackage{pgf-umlsd}
\usepackage{ifthen}
\usepackage{graphicx}
\usepackage{hyperref}
\usepackage{cleveref}


\usepackage[ruled,longend,linesnumbered]{algorithm2e}
%
%\RequirePackage{fix-cm}

\usepackage{pgfplots}
% \usepackage{fancyhdr}
% \pagesty

\usepackage{xcolor}
\newcommand{\todo}[1]{{\color{blue}[TODO: #1]}}
\newcommand{\response}[1]{{\color{green}[RESPONSE: #1]}}

\title{\LARGE \bf
	Evolutionary Nonlinear Model Predictive Control on a \\ Fixed-Wing and Quadrotor with Uncertainty
}

%\titlerunning{Short form of title}        % if too long for running head

\author{Jaron Ellingson$^{1}$, Mat Haskell$^{2}$%
%	\thanks{*This material is based upon work supported by the National Science Foundation under Grant No. 1727010. This work is also supported by the Center for Unmanned Aircraft Systems (C-UAS), a National Science Foundation Industry/University Cooperative Research Center (I/UCRC) under NSF award No. IIP-1161036 along with significant contributions from C-UAS industry members.}% <-this % stops a space
	\thanks{$^{1}$ Jaron Ellingson is an MS candidate in the Department of Mechanical Engineering, Brigham Young University
		{\tt\small jaronce@byu.eduu}}%
	\thanks{$^{2}$ Mat Haskell is a PhD candidate in the Department of Mechanical Engineering, Brigham Young University
		{\tt\small mathew.haskell@byu.edu}}%
}

\begin{document}
\maketitle
\thispagestyle{empty}
\pagestyle{empty}

\date{Received: date / Accepted: date}
% The correct dates will be entered by the editor


\maketitle

%\begin{abstract}
%Abstract

% \PACS{PACS code1 \and PACS code2 \and more}
% \subclass{MSC code1 \and MSC code2 \and more}
%\end{abstract}

\section{Project Direction}

Real-time model predictive control (MPC) is limited to short time horizons and linear systems because the optimization complexity is too large with long time horizons and nonlinear systems. For this reason, MPC is typically accomplished using linearized models and convex optimization solvers. We seek to explore evolutionary algorithms allowing for nonlinear models and constraints, non-convex costs, and extended time horizons. An evolutionary algorithm can be parallelized because each state propagation is independent of the others.
While parallelization of the evolutionary algorithm allows for real-time control, writing the code to parallelize the algorithm might be out of the scope for this project. A neural network is used in the NEMPC algorithm in \cite{hyatt2019real}, which interfaced with the GPU under the hood. This work seeks to implement NEMPC without using a neural network, making parallelization more difficult. This work is a continuation of the work accomplished in \cite{hyatt2020parameterized} which is developed to parameterize long time horizons. We also intend to parameterize the design space of the optimization to reduce solve times. All previous work using evolutionary MPC was performed on robotic arms. We seek to apply these algorithms to control UAVs, both fixed-wing and multirotor.


\section{Background}

In \cite{hyatt2017real}, linear MPC was performed in real-time on a robotic arm using an evolutionary optimization algorithm rather than a typical convex solver. To make the algorithm real-time, each state propagation in the evolutionary algorithm during a single generation occured in parallel on a graphics processing unit (GPU). GPU's are known to be fast at matrix multiplications, which the linearized model provided.

The previous work using an evolutionary algorithm was continued in \cite{hyatt2019real} where nonlinear dynamics were used as the model for real-time MPC. The key change in this work from \cite{hyatt2017real} is that the nonlinear dynamics were approximated using a neural network. The neural network was able to learn the nonlinear dynamics in a way that still utilized matrix multiplications, allowing for parallelization of the genetic algorithm on a GPU.

In \cite{hyatt2020parameterized}, the evolutionary MPC algorithm is compared to a MPC using a QP solver both with a parameterized optimization design space to reduce the number of design variables. Both algorithms are capable of real-time control of nonlinear robotic arms. With MPC, the optimization design variable is the future trajectory of control inputs that should be applied to the system over a finite time horizon. This work used a piece-wise linear function to parameterize the trajectory of future inputs, usually with only 2 lines (3 points). This allowed for a significant reduction in the search space of the optimization and thus, faster solve times. With the parameterization, the solve times of both algorithms are capable of running control at over 100 Hz. MPC using the QP solver was still faster than the parallelized NEMPC, but the evolutionary algorithm allows for a nonlinear model.

\section{Team Roles}

Mat and Jaron will both work on the evolutionary algorithm. Mat will apply it to a quadrotor while Jaron will apply it to a fixed-wing aircraft. 

\bibliographystyle{unsrt}
\bibliography{references} % bibliography data in references.bib
\bibliographystyle{IEEEtran}

\end{document}
