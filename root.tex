\documentclass[letterpaper, 10 pt, conference]{ieeeconf}  % Comment this line out if you need a paper

%\documentclass[a4paper, 10pt, conference]{ieeeconf}      % Use this line for a4 paper

\IEEEoverridecommandlockouts

\overrideIEEEmargins                                      % Needed to meet printer requirements.

% See the \addtolength command later in the file to balance the column lengths
% on the last page of the document
\usepackage{cite}
% The following packages can be found on http:\\www.ctan.org
\usepackage{graphicx} % for pdf, bitmapped graphics files
%\usepackage{epsfig} % for postscript graphics files
%\usepackage{mathptmx} % assumes new font selection scheme installed
%\usepackage{times} % assumes new font selection scheme installed
\usepackage{amsmath} % assumes amsmath package installed
\usepackage{amssymb}  % assumes amsmath package installed
\usepackage{math}
\usepackage{aircraftshapes}
%\usepackage{natbib}
%\usepackage{standalone}
\usepackage{tikz}
\usetikzlibrary{positioning}
\usetikzlibrary{shapes,arrows}
\tikzstyle{pre}=[<-,>=stealth,thick]
\tikzstyle{post}=[->,>=stealth,thick]
\tikzstyle{prepost}=[<->,>=stealth,thick]
\usetikzlibrary{automata,positioning,graphs,calc}

\usepackage{graphicx}
\usepackage{color}
\usepackage{pgfplots}
\pgfplotsset{compat=1.15}
\usepackage{pgf-umlsd}
\usepackage{ifthen}
\usepackage{graphicx}
\usepackage{hyperref}
\usepackage{cleveref}


\usepackage[ruled,longend,linesnumbered]{algorithm2e}
%
%\RequirePackage{fix-cm}

\usepackage{pgfplots}
% \usepackage{fancyhdr}
% \pagesty

\usepackage{xcolor}
\newcommand{\todo}[1]{{\color{blue}[TODO: #1]}}
\newcommand{\response}[1]{{\color{green}[RESPONSE: #1]}}

\title{\LARGE \bf
	Evolutionary Nonlinear Model Predictive Control on a \\ Fixed-Wing and Quadrotor with Uncertainty
}

%\titlerunning{Short form of title}        % if too long for running head

\author{Jaron Ellingson$^{1}$, Mat Haskell$^{2}$%
	\thanks{*This material is based upon work supported by the National Science Foundation under Grant No. 1727010. This work is also supported by the Center for Unmanned Aircraft Systems (C-UAS), a National Science Foundation Industry/University Cooperative Research Center (I/UCRC) under NSF award No. IIP-1161036 along with significant contributions from C-UAS industry members.}% <-this % stops a space
	\thanks{$^{1}$ Jaron Ellingson is an MS candidate in the Department of Mechanical Engineering, Brigham Young University
		{\tt\small jaronce@byu.eduu}}%
	\thanks{$^{2}$ Mat Haskell is a PhD candidate in the Department of Mechanical Engineering, Brigham Young University
		{\tt\small email@byu.edu}}%
}

\begin{document}
\maketitle
\thispagestyle{empty}
\pagestyle{empty}

\date{Received: date / Accepted: date}
% The correct dates will be entered by the editor


\maketitle

%\begin{abstract}
%Abstract

% \PACS{PACS code1 \and PACS code2 \and more}
% \subclass{MSC code1 \and MSC code2 \and more}
%\end{abstract}

\section{Project Direction}

Model predictive control is limited to short time horizons and linear systems because the optimization complexity is too large with long time horizons and nonlinear systems. We seek to explore evolutionary algorithms that will allow for complex model predictive control. This work is a continuation of the work accomplished in \todo{cite Phil's nonlinear EMPC} which is developed to parameterize long time horizons. This along with an efficient evolutionary algorithm seeks to decrease the optimization search space while also approximating the underlying model. We believe that the work in \todo{phil's work} did not fully exercise the benefits of longer time horizons or introducing random genetics into the population. Furthermore, this work did not take model uncertainty analysis into consideration. We believe an evolutionary model can find a model which is robust to disturbances through this analysis. 	


\section{Background}

(linear) Real-time evolutionary model predictive control using a graphics processing unit \cite{hyatt2017real}. 

\section{Team Roles}

Mat and Jaron will both work on the evolutionary algorithm. Mat will apply it to a quadrotor while Jaron will apply it to a fixed-wing aircraft. 

\bibliographystyle{unsrt}
\bibliography{references} % bibliography data in references.bib
\bibliographystyle{IEEEtran}

\end{document}
